\chapter{Installation}

\section{Prerequisites}

Moa is developed on Ubuntu \citep{ubuntu} and RHEL \citep{rhel} Linux
and is expected to operate without much problems on most modern Linux
distributions. Moa is depends on the following software:\footnote{The
  version numbers are an indication of the version used by the author,
  not strict prerequisites. Other, even older, versions might work}

\begin{itemize}
\item Gnu Make 3.81 \citep{Gnumake}
\item Git 1.6 \citep{git}.
\item Python 2.6 \citep{python}\footnote{Python 2.5 will not
  work. Several supporting scripts use 2.6 specific functionality}
\item Apache Couchdb 0.9.0 \citep{couchdb}\footnote{Only if using
  couchdb functionality \label{fn:usingcouchdb}}
\item Biopython 1.49 \citep{biopython}
\item Couchdb-python \citep{couchpy}\footref{fn:usingcouchdb}
\end{itemize}

Furthermore, the bioinformatics analysis tools that are likely to be
used need to be installed. All Moa templates that wrap an application
expect that application to be installed and present in the PATH.
For more information see section \ref{s:installBTools}.

\section{Getting the code}

Moa is hosted at github:

\begin{lsturl}
http://github.com/mfiers/Moa
\end{lsturl}

Currently there are no formal releases so the only option is to
download the latest version of the software, eiter using git \citep{git}:

\begin{bash}
git clone git://github.com/mfiers/Moa.git
\end{bash}

It is also possible to download an (automatically generated) archive
of the trunk. Using Git, however, makes it very easy to stay in sync
with the latest bugfixes and is thus strongly recommended until there
are formal releases. An archive can be found here:

\begin{bash}
http://github.com/mfiers/Moa/tarball/master
\end{bash}

After downloading, and possibly unpacking, the source code must be
moved to a suitable location of your choice. For example
\lstinline!/opt/moa!. The resulting tree should contain the following
directories: \lstinline!/opt/moa/bin! and
\lstinline!/opt/moa/template!. Remember to properly set the rights on
the files, depending on who is going to use the software.

\section{Configuration}

Configuration of Moa is fairly simple, 

