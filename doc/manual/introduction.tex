\chapter{Introduction}

Moa is a piece of software build around GNU Make \citep{Gnumake} that
allows you to use Gnu Make run bioinformatics pipelines. 

GNU Make is an excellent tool to automate the compilation of
software. Gnu make determines how a file is created, what it's
dependencies are, and what needs to be executed. Gnu Make uses so
called Makefiles to describe a project. A bioinformatics project is
often of the same form as compiling software.

Moa wraps a set of common bioinformatics tools as Makefiles. Features
of Moa are:

\begin{itemize}
\item Uniform interface. All Moa makefiles use a library that provides
  a uniform, command line, interface to configuring and executing jobs.
\item Interaction. The makefiles are designed to interact with each
  other.
\item Parallel execution. 
\end{itemize}

Moa consists of several parts:

\begin{itemize}
\item moaBase. A central library describing a number of central
  routines used by all Makefiles
\item template Makefiles. Each application is wrapped in a template
  Makefile. 
\item The ``moa'' script. A number of tools that cannot be caught in
  Makefiles are implemented in a cental helper script called ``moa''.
\item Helper scripts. A number of diverse utilities are part of the
  moa packages. These are a part of an embedded application.
\item Couchdb interface. Moa is able to store information on each job
  in a couchdb. See chapter XX.
\end{itemize}

To better understand how Moa works, please read this sample session:

\begin{bash}
mkdir test
cd test
moa new lft
\end{bash}



% LocalWords:  conf makefiles Moa moaBase
