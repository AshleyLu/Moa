\documentclass[a4paper,11pt,oneside]{article}

\renewcommand{\familydefault}{\sfdefault}
\usepackage[super]{natbib}

\begin{document}
\title{Annotation of the potato genome using GNU-make}
\author{Mark Fiers, Susan Thomson, Jeanne Jacobs}

\section{Introduction}

An increase in genome sequencing capacity requires a matching increase in
the capability to analyse the sequence. Plant \& Food Research is part
of the international Potato Genome Sequencing Consortium (PGSC). To
make optimal use of the sequence it needs to be annotated as it is
generated. 

Genome annotation is a complex operation. Multiple interdependent
analyses steps need to be executed on many, often thousands, different
objects. There is a multitude of tools around that assist in the
process of genome annotation. Many of these tools, however, have one
or more drawbacks concerning installation, capacity and flexibility.

Particularly flexibility is a major issue as almost each experiment
has differing analysis requirements. A common solution is to develop
custom scripts that handle (a part of) the analysis pipeline. Such
scripts should, ideally, be an integral part of the pipeline,
interacting with other, more standardized, building blocks.

\section{GNU Make and Moa}

In this poster we describe how to create flexible pipelines, able to
embed customized scripts and capable of processing large amounts of
data. We build these pipelines using a widely available piece of
software: ``GNU Make''.

Gnu Make is a ubiquitous tool that aids in compiling software.
Compilation requires the processing of many interdependent source
files. The compilation process is described in a ``Makefile'',
sequently used by GNU Make to automate the process.

We have created a set of generic Makefiles. The Makefiles describe
common operations in genome annotation (amongst others: Blast, Blat,
Glimmer, GMap and Bowtie). The Makefiles are developed to closely work
together as building blocks in an annotation pipeline. The structure
of the Makefiles allows for easy embedding of custom scripts and,
hence, the ability to generate very flexible pipelines. The combined
software is called Moa.

By careful design, and by making optimal use of GNU Make, the Moa software
is capable of:
 
\begin{itemize}
\item Seamlessly chain building blocks. All Makefiles are written to
  make use of the output of other components.
\item Parallel execution of multiple jobs, faciliated by GNU Make.
\item Embedding custom scripts as an integral part of a pipeline.
\item Repeat only those jobs what are necessary (after a data update).
\item Tracking provenance. The pipeline structure and all intermediate
  data are stored as files, making them easily accessible.
\item Upload data to a Generic Genome Browser database for
  visualization.
\end{itemize}

\section{Usage}

The software aims at providing a bioinformaticist a framework that
facilitates the building of annotation pipelines without losing
flexibility. All Makefiles use an underlying library that provides a
uniform command line interface to using the software. Figure 1 shows a
short session that shows how to use Moa.

\section{Application}

Using these Makefiles we have build a pipeline (see Figure 2) that
annotates 1300+ potato BAC sequences. It is possible to reexecute the
pipeline, or parts of it, after a set interval. 

\section{Captions}

\textbf{Figure 1} This figure show a small session on how to run a Moa pipeline. 


\textbf{Figure 2}: A representation of (a part of) the current potato
genome annotation pipeline.

%\bibliographystyle{plain}
%\bibliography{poster}
\end{document}

% LocalWords:  internet Ensembl bioinformaticians Makefile workflow workflows
% LocalWords:  bioinformatics Makefiles Bowtie BAC
